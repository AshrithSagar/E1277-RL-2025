\section*{Question 3}

We have so far discussed infinite-horizon reinforcement learning with exponential discounting.
In practice, one is also interested in quasi-hyperbolic discounting.
In this case, we have two parameter \( \sigma, \gamma \in [0,1) \), and the value function \( J_{\mu} \in \mathbb{R}^{\vert\mathcal{S}\vert} \) of a stationary (possibly stochastic) policy \( \mu \) is given by
\[
    J_{\mu}(s)=\mathbb{E}\left[\sum_{n=0}^{\infty} d_{n} g\left(s_{n}, a_{n}, s_{n+1}\right) \mid s_{0}=s\right],
\]
where
\[
    d_{n}= \begin{cases}1 & \text { if } n=0 \\ \sigma \gamma^{n} & \text { if } n \geq 1\end{cases}
\]
further, \( a_{n} \sim \mu\left(\cdot \mid s_{n}\right) \) and \( s_{n+1} \sim \mathcal{P}\left(\cdot \mid s_{n}, a_{n}\right) \) for \( n \geq 0 \).

Answer the following questions.
\begin{enumerate}[label= (\alph*), noitemsep]
    \item Does there exist a Bellman-type relation for \( J_{\mu} \)?
    \item From the above relation, can you identify the Bellman operator \( T_{\mu} \)?
          Is this operator a contraction?
    \item Suppose the transition matrix \( \mathcal{P} \) is unknown.
          Can you design a model-free algorithm to estimate \( J_{\mu} \)?
          You can presume that you can sample from the invariant distribution of the Markov chain induced by \( \mu \).
          Discuss the almost convergence of this algorithm.
          You may directly use the results that were discussed in class.
\end{enumerate}

\subsection*{Solution}
