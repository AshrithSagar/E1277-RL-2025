\section*{Question 2}

Let \( T_{\mu} \) be the Bellman operator defined in the above question. \\
Show that \( T_{\mu} \) is a \( \gamma \)--contraction with respect to \( {\Vert \cdot \Vert}_{D_{\mu}} \).

\subsection*{Solution}

We need to show that \( \Vert T_{\mu} J_{1} - T_{\mu} J_{2} \Vert_{D_{\mu}} \leq \gamma \Vert J_{1} - J_{2} \Vert_{D_{\mu}} \) for all \( J_{1}, J_{2} \in \mathbb{R}^{S} \).

Starting with the left-hand side, we have
\begin{align*}
    \Vert T_{\mu} J_{1} - T_{\mu} J_{2} \Vert_{D_{\mu}}
     & =
    \Vert r_{\mu} + \gamma P_{\mu} J_{1} - r_{\mu} - \gamma P_{\mu} J_{2} \Vert_{D_{\mu}}
    \\
     & =
    \Vert \gamma P_{\mu} J_{1} - \gamma P_{\mu} J_{2} \Vert_{D_{\mu}}
    \\
     & =
    \gamma \Vert P_{\mu} J_{1} - P_{\mu} J_{2} \Vert_{D_{\mu}}
    \\
     & \leq
    \gamma \Vert J_{1} - J_{2} \Vert_{D_{\mu}}
\end{align*}
where the last step follows from the definition of the \( D_{\mu} \) norm, which is defined as \( \Vert J \Vert_{D_{\mu}} = \sqrt{J^{\top} D_{\mu} J} \), and the fact that \( P_{\mu} \) is a contraction mapping.
Thus, we have shown that \( T_{\mu} \) is a \( \gamma \)--contraction with respect to \( {\Vert \cdot \Vert}_{D_{\mu}} \):
\[
    \Vert T_{\mu} J_{1} - T_{\mu} J_{2} \Vert_{D_{\mu}} \leq \gamma \Vert J_{1} - J_{2} \Vert_{D_{\mu}}
\]
This completes the proof.
