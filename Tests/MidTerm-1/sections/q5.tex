\section*{Question 5}

Consider a mapping \( W: \mathbb{R}^{\vertS\vert} \rightarrow \mathbb{R}^{\vertS\vert} \) that is not a contraction.
However, a \( q \)-times composition of \( W \) with itself is a contraction for some \( q>1 \).
In other words, \( W^{q} \): \( \mathbb{R}^{\vertS\vert} \rightarrow \mathbb{R}^{\vertS\vert} \) is a contraction.
Show the following:
\begin{enumerate}[noitemsep]
    \item There exists a fixed point of \( W \).
    \item The above fixed point of \( W \) is unique.
\end{enumerate}

\subsection*{Solution}

\subsubsection*{Existence of a fixed point of \( W \)}

Since \( \mathbb{R}^{\vertS\vert} \) is a complete metric space, and \( W^{q} \) is a contraction mapping with \( q > 1 \) there exists a unique fixed point \( x^{*} \) of \( W^{q} \), by the contraction mapping theorem, thereby \( W^{q}(x^{*}) = x^{*} \).
Now, we have
\[
    W^{q+1}(x^{*}) = W^{q}(W(x^{*})) = W(W^{q}(x^{*})) = W(x^{*})
\]
which implies that \( W(x^{*}) \) is also a fixed point of \( W^{q} \).
Since the fixed point of \( W^{q} \) is unique, we have \( W(x^{*}) = x^{*} \), which implies that \( x^{*} \) is a fixed point of \( W \).

\subsubsection*{Uniqueness of the fixed point of \( W \)}

Can be shown through proof by contradiction.
Assume that there exists another fixed point \( y^{*} \) of \( W \) such that \( y^{*} \neq x^{*} \).
Upon applying \( W \) to both these points \( q \) times, we get
\[
    W^{q}(x^{*}) = x^{*} \quad \text{and} \quad W^{q}(y^{*}) = y^{*}
\]
This implies that \( x^{*} \) and \( y^{*} \) are fixed points of \( W^{q} \) as well.
But since \( W^{q} \) has a unique fixed point, we have \( x^{*} = y^{*} \), which is a contradiction.
Hence, the fixed point of \( W \) is unique, which is the same as the fixed point of \( W^{q} \).
