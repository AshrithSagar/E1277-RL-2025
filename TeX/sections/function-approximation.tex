\section{Function approximation}

In general, taking examples from a desired function and attempting to generalize from them to construct an approximation of the entire function.

The approximate value function \( \hat v(s; \mathbf{w}) \approx v_\pi(s) \) is represented not as a table but as a parameterized functional form with weight vector \( \mathbf{w} \in \mathbb{R}^d \), with \( d \ll \vert \mathcal{S} \vert \).

The \textit{mean square value error} \( \overline{\text{VE}} \) is defined as
\begin{equation}
    \overline{\text{VE}}(\mathbf{w}) \doteq \sum_{s \in \mathcal{S}} \mu(s) {\Big[ v_\pi(s) - \hat v(s; \mathbf{w}) \Big]}^2
\end{equation}
where \( \mu(s) \) is some state distribution, \( \mu(s) \geq 0, \ \sum_{s \in \mathcal{S}} \mu(s) = 1 \), representing how much we care about the error in each state \( s \).

\( \eta(s) \to \) Number of time steps spent on average in a state \( s \) in a single episode.
\begin{equation}
    \mu(s) = \frac{\eta(s)}{\sum_{s'} \eta(s')}
    , \quad \forall s \in \mathcal{S}
\end{equation}

\subsection{Stochastic gradient descent (SGD)}

\begin{equation}
    \begin{aligned}
        \mathbf{w}_{t+1}
         & \doteq
        \mathbf{w}_t - \frac{1}{2} \alpha \nabla {\Big[ v_\pi(S_t) - \hat v(S_t; \mathbf{w}_t) \Big]}^2
        \\ & =
        \mathbf{w}_t + \alpha \Big[ v_\pi(S_t) - \hat v(S_t; \mathbf{w}_t) \Big] \nabla \hat v(S_t; \mathbf{w}_t)
        , \quad \alpha > 0
    \end{aligned}
\end{equation}

\subsection{Linear function approximation}
