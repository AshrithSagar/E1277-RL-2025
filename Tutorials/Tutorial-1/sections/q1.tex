\section*{Question 1}

Consider a \( k \)-armed bandit problem with \( k = 4 \) actions, denoted 1, 2, 3, and 4.
Consider applying to this problem a bandit algorithm using \( \epsilon \)-greedy action selection, sample-average action-value estimates, and initial estimates of \( Q_{1}(a) = 0 \), for all \( a \).
Suppose the initial sequence of actions and rewards is \( A_{1} = 1, R_{1} = 1, A_{2} = 2, R_{2} = 1, A_{3} = 2, R_{3} = 2, A_{4} = 2, R_{4} = 2, A_{5} = 3, R_{5} = 0 \).
On some of these time steps the \( \epsilon \) case may have occurred, causing an action to be selected at random.
On which time steps did this definitely occur?
On which time steps could this possibly have occurred?

\subsection*{Solution}

The \( \epsilon \)-greedy action selection algorithm selects a random action with probability \( \epsilon \) which is the exploration case and selects the greedy action with probability \( 1 - \epsilon \) which is the exploitation case.
The \( Q \) values by the sample-average action-value estimates are defined as
\[
    Q_{t}(a) = \frac{\text{sum of rewards when action \( a \) taken prior to time \( t \)}}{\text{number of times action \( a \) taken prior to time \( t \)}}
    =
    \frac{\sum_{i=1}^{t-1} R_{i} \cdot \mathbb{I}(A_{i} = a)}{\sum_{i=1}^{t-1} \mathbb{I}(A_{i} = a)}
\]
where \( \mathbb{I}(A_{i} = a) \) is the indicator function which is 1 if \( A_{i} = a \) and 0 otherwise.
We're given the initial estimates of \( Q_{1}(a) = 0 \) for all \( a \).

The sequence of actions and rewards is given as
\section*{Question 1}

Consider the MDP \( \mathcal{M} \equiv(\mathcal{S}, \mathcal{A}, P, r, \gamma) \) with \( \vert \mathcal{S} \vert = S \) and \( \vert \mathcal{A} \vert = A \).
Suppose \( \mu \) is a stochastic policy and \( \Phi \in \mathbb{R}^{S \times d} \) a feature matrix for some \( d \geq 1 \).
Let \( P_{\mu} \) be the \( S \times S \) matrix given by
\[
    P_{\mu}\left(s^{\prime} \mid s\right)=\sum_{a} \mu(a \mid s) P\left(s^{\prime} \mid s, a\right)
\]
This matrix represents the transition matrix of the Markov chain \( \left(\mathcal{S}, P_{\mu}\right) \) induced by \( \mu \).
Suppose this Markov chain is ergodic so that it has a unique stationary distribution, which we denote by \( d_{\mu} \).
Let \( D_{\mu} \) be the \( S \times S \) diagonal matrix whose diagonal is \( d_{\mu} \), and let
\[
    A=\Phi^{\top} D_{\mu}\left(\mathbb{I}-\gamma P_{\mu}\right) \Phi \quad \text { and } b=\Phi^{\top} D_{\mu} r_{\mu},
\]
where \( r_{\mu}(s)=\sum_{a} \mu(a \mid s) r(s, a) \).
Let \( \Pi: \mathbb{R}^{S} \rightarrow \mathbb{R}^{S} \) be given by \( \Pi J=\Phi{\left(\Phi^{\top} D_{\mu} \Phi\right)}^{-1} \Phi^{\top} D_{\mu} J \).

Show that \( \theta_\ast:=A^{-1} b \) is the fixed point of the projected Bellman operator, i.e., \( \Pi T_{\mu} \Phi \theta_\ast=\Phi \theta_\ast \), where \( T_{\mu}: \mathbb{R}^{S} \rightarrow \mathbb{R}^{S} \) is the Bellman operator satisfying \( T_{\mu} J=r_{\mu}+\gamma P_{\mu} J \).

\subsection*{Solution}

Given an MDP \( \mathcal{M} \equiv(\mathcal{S}, \mathcal{A}, P, r, \gamma) \) with \( \vert \mathcal{S} \vert = S \) and \( \vert \mathcal{A} \vert = A \), we have the Bellman operator defined as
\[
    T_{\mu} J(s) = r_{\mu}(s) + \gamma \sum_{s^{\prime}} P_{\mu}(s^{\prime} \mid s) J(s^{\prime}),
\]
where \( J \in \mathbb{R}^{S} \) is the value function.

In (linear) stochastic function approximation, we are interested in finding a \( \theta_\ast \in \mathbb{R}^{d} \) such that \( J_\mu \approx \Phi \theta_\ast \), where \( J_\mu \) is the optimal value function for the policy \( \mu \), and \( \Phi \in \mathbb{R}^{S \times d} \) is a feature matrix.
The reason we want to do this is because the dimension of \( J_\mu \) is \( S \), which can be very large, in general, and we want to approximate it with a smaller dimensional representation \( \Phi \theta_\ast \).
From general MDP theory, we know that the optimal value function \( J_\mu \), for a given policy \( \mu \), satisfies the Bellman equation
\[
    J_\mu(s) = r_{\mu}(s) + \gamma \sum_{s^{\prime}} P_{\mu}(s^{\prime} \mid s) J_\mu(s^{\prime}),
    \implies
    J_\mu = r_{\mu} + \gamma P_{\mu} J_\mu.
    \implies
    \left(\mathbb{I}-\gamma P_{\mu}\right) J_\mu = r_{\mu}.
\]
Given that the underlying Markov chain \( \left(\mathcal{S}, P_{\mu}\right) \) induced by \( \mu \) is ergodic, it converges to a unique stationary distribution \( d_{\mu}: \mathcal{S} \rightarrow \mathbb{R} \) such that
\[
    d_{\mu}(s) = \sum_{s^{\prime}} P_{\mu}(s^{\prime} \mid s) d_{\mu}(s^{\prime}),
    \implies
    d_{\mu} = P_{\mu}^{\top} d_{\mu}.
\]

Since we want a \( \theta_\ast \) such that \( J_\mu \approx \Phi \theta_\ast \), we start by posing this as an optimization problem, i.e., we want to find some \( \theta \) such that
\[
    f(\theta) = \frac{1}{2} \Vert J_\mu - \Phi \theta \Vert_{D_{\mu}}^2
    = \frac{1}{2} \sum_{s} d_{\mu}(s) {\left[ J_\mu(s) - \phi^\top(s) \theta \right]}^2
\]
is minimized.
This is how we define the optimality for how we want to approximate the value function \( J_\mu \) with a smaller dimensional representation \( \Phi \theta \).
For such a function \( f(\theta) \), we can find the optimal \( \theta_\ast \) by setting the gradient of \( f(\theta) \) to zero, i.e.,
\[
    \nabla f(\theta_\ast) = - \sum_{s} d_{\mu}(s) \left[ J_\mu(s) - \phi^\top(s) \theta_\ast \right] \phi(s) = 0
\]
\[
    \implies
    \sum_{s} d_{\mu}(s) J_\mu(s) \phi(s) = \sum_{s} d_{\mu}(s) {\left[ \phi^\top(s) \theta_\ast \right]} \phi(s)
\]
This is the projected Bellman equation, we're interested to find.
In matrix form, we have
\begin{align*}
    \text{LHS}
     & \to
    \underbrace{\Phi^\top}_{d \times S} \overbrace{D_\mu}^{S \times S} \underbrace{J_\mu}_{S \times 1}
    \\
    \text{RHS}
     & \to
    \sum_{s} d_\mu(s) \phi(s) \phi^\top(s) \theta = \underbrace{\Phi^\top}_{d \times S} \overbrace{D_\mu}^{S \times S} \underbrace{\Phi}_{S \times d} \overbrace{\theta}^{d \times 1}
\end{align*}
\begin{align*}
    \implies
    \Phi^\top D_\mu J_\mu
     & =
    \Phi^\top D_\mu \Phi \theta_\ast
    \\
    \implies
    \theta_\ast
     & =
    {(\Phi^\top D_\mu \Phi)}^{-1} \Phi^\top D_\mu J_\mu
    \\
    \implies
    \Phi \theta_\ast
     & =
    \Phi {(\Phi^\top D_\mu \Phi)}^{-1} \Phi^\top D_\mu J_\mu
\end{align*}

This such \( \Phi \theta_\ast \) is closest approximation to \( J_\mu \) in the column space of \( \Phi \), \(\operatorname{col}(\Phi) \).

Now, we can appropriately define the project Bellman operator as
\[
    \Pi J = \Phi {(\Phi^\top D_\mu \Phi)}^{-1} \Phi^\top D_\mu J
\]

Starting with the Bellman equation, and applying the projected Bellman operator \( \Pi \) on both sides, and using \( J = \Phi \theta_\ast \), we have
\begin{align*}
    T_{\mu} J
     & =
    r_{\mu} + \gamma P_{\mu} J
    \\
    \implies
    \Pi T_{\mu} J
     & =
    \Pi \left(r_{\mu} + \gamma P_{\mu} J\right)
    =
    \Phi{\left(\Phi^{\top} D_{\mu} \Phi\right)}^{-1} \Phi^{\top} D_{\mu} J
    \\
    \implies
    \Pi T_{\mu} \Phi \theta_\ast
     & =
    \Phi{\left(\Phi^{\top} D_{\mu} \Phi\right)}^{-1} \Phi^{\top} D_{\mu} \Phi \theta_\ast
    \\ & =
    \Phi \cancel{{\left(\Phi^{\top} D_{\mu} \Phi\right)}^{-1}} \cancel{\left(\Phi^{\top} D_{\mu} \Phi\right)} \theta_\ast
    =
    \Phi \theta_\ast
    \\
    \therefore
    \Pi T_{\mu} \Phi \theta_\ast
     & =
    \Phi \theta_\ast
\end{align*}
as required, showing that \( \theta_\ast \) is the fixed point of the projected Bellman operator.

Starting with \( A \theta_\ast = b \), and substituting the definitions of \( A \) and \( b \), we have
\begin{align*}
    \implies
    \Phi^{\top} D_{\mu}\left(\mathbb{I}-\gamma P_{\mu}\right) \Phi \theta_\ast
     & =
    \Phi^{\top} D_{\mu} r_{\mu}
    \\
    \implies
    \Phi^{\top} D_{\mu} \Phi \theta_\ast - \gamma \Phi^{\top} D_{\mu} P_{\mu} \Phi \theta_\ast
     & =
    \Phi^{\top} D_{\mu} r_{\mu}
    \\
    \implies
    \Phi^{\top} D_{\mu} \Phi \theta_\ast
     & =
    \Phi^{\top} D_{\mu} r_{\mu} + \gamma \Phi^{\top} D_{\mu} P_{\mu} \Phi \theta_\ast
    \\ & =
    \Phi^{\top} D_{\mu} \left( r_{\mu} + \gamma P_{\mu} \Phi \theta_\ast \right)
    =
    \Phi^{\top} D_{\mu} T_{\mu} \Phi \theta_\ast
\end{align*}

\begin{align*}
    \Pi T_{\mu} \Phi \theta_\ast
     & =
    \Pi \left( r_{\mu} + \gamma P_{\mu} \Phi \theta_\ast \right)
    =
    \Phi{\left(\Phi^{\top} D_{\mu} \Phi\right)}^{-1} \Phi^{\top} D_{\mu} \left( r_{\mu} + \gamma P_{\mu} \Phi \theta_\ast \right)
    \\ & =
    \Phi{\left(\Phi^{\top} D_{\mu} \Phi\right)}^{-1} \Phi^{\top} D_{\mu} r_{\mu} + \gamma \Phi{\left(\Phi^{\top} D_{\mu} \Phi\right)}^{-1} \Phi^{\top} D_{\mu} P_{\mu} \Phi \theta_\ast
\end{align*}

