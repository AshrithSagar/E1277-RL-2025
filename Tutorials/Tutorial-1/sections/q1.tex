\section*{Question 1}

Consider a \( k \)-armed bandit problem with \( k = 4 \) actions, denoted 1, 2, 3, and 4.
Consider applying to this problem a bandit algorithm using \( \epsilon \)-greedy action selection, sample-average action-value estimates, and initial estimates of \( Q_{1}(a) = 0 \), for all \( a \).
Suppose the initial sequence of actions and rewards is \( A_{1} = 1, R_{1} = 1, A_{2} = 2, R_{2} = 1, A_{3} = 2, R_{3} = 2, A_{4} = 2, R_{4} = 2, A_{5} = 3, R_{5} = 0 \).
On some of these time steps the \( \epsilon \) case may have occurred, causing an action to be selected at random.
On which time steps did this definitely occur?
On which time steps could this possibly have occurred?

\subsection*{Solution}

The \( \epsilon \)-greedy action selection algorithm selects a random action with probability \( \epsilon \) which is the exploration case and selects the greedy action with probability \( 1 - \epsilon \) which is the exploitation case.
The \( Q \) values by the sample-average action-value estimates are defined as
\[
    Q_{t}(a) = \frac{\text{sum of rewards when action \( a \) taken prior to time \( t \)}}{\text{number of times action \( a \) taken prior to time \( t \)}}
    =
    \frac{\sum_{i=1}^{t-1} R_{i} \cdot \mathbb{I}(A_{i} = a)}{\sum_{i=1}^{t-1} \mathbb{I}(A_{i} = a)}
\]
where \( \mathbb{I}(A_{i} = a) \) is the indicator function which is 1 if \( A_{i} = a \) and 0 otherwise.
We're given the initial estimates of \( Q_{1}(a) = 0 \) for all \( a \).

The sequence of actions and rewards is given as
\section*{Question 1}

Consider the MDP \( \mathcal{M} \equiv(\mathcal{S}, \mathcal{A}, P, r, \gamma) \) with \( \vert \mathcal{S} \vert = S \) and \( \vert \mathcal{A} \vert = A \).
Suppose \( \mu \) is a stochastic policy and \( \Phi \in \mathbb{R}^{S \times d} \) a feature matrix for some \( d \geq 1 \).
Let \( P_{\mu} \) be the \( S \times S \) matrix given by
\[
    P_{\mu}\left(s^{\prime} \mid s\right)=\sum_{a} \mu(a \mid s) P\left(s^{\prime} \mid s, a\right)
\]
This matrix represents the transition matrix of the Markov chain \( \left(\mathcal{S}, P_{\mu}\right) \) induced by \( \mu \).
Suppose this Markov chain is ergodic so that it has a unique stationary distribution, which we denote by \( d_{\mu} \).
Let \( D_{\mu} \) be the \( S \times S \) diagonal matrix whose diagonal is \( d_{\mu} \), and let
\[
    A=\Phi^{\top} D_{\mu}\left(\mathbb{I}-\gamma P_{\mu}\right) \Phi \quad \text { and } b=\Phi^{\top} D_{\mu} r_{\mu},
\]
where \( r_{\mu}(s)=\sum_{a} \mu(a \mid s) r(s, a) \).
Let \( \Pi: \mathbb{R}^{S} \rightarrow \mathbb{R}^{S} \) be given by \( \Pi J=\Phi{\left(\Phi^{\top} D_{\mu} \Phi\right)}^{-1} \Phi^{\top} D_{\mu} J \).

Show that \( \theta_{*}:=A^{-1} b \) is the fixed point of the projected Bellman operator, i.e., \( \Pi T_{\mu} \Phi \theta_{*}=\Phi \theta_{*} \), where \( T_{\mu}: \mathbb{R}^{S} \rightarrow \mathbb{R}^{S} \) is the Bellman operator satisfying \( T_{\mu} J=r_{\mu}+\gamma P_{\mu} J \).

\subsection*{Solution}

Starting with the Bellman equation, and applying the projected Bellman operator \( \Pi \) on both sides, and using \( J = \Phi \theta_{*} \), we have
\begin{align*}
    T_{\mu} J
     & =
    r_{\mu} + \gamma P_{\mu} J
    \\
    \implies
    \Pi T_{\mu} J
     & =
    \Pi \left(r_{\mu} + \gamma P_{\mu} J\right)
    =
    \Phi{\left(\Phi^{\top} D_{\mu} \Phi\right)}^{-1} \Phi^{\top} D_{\mu} J
    \\
    \implies
    \Pi T_{\mu} \Phi \theta_{*}
     & =
    \Phi{\left(\Phi^{\top} D_{\mu} \Phi\right)}^{-1} \Phi^{\top} D_{\mu} \Phi \theta_{*}
    \\ & =
    \Phi \cancel{{\left(\Phi^{\top} D_{\mu} \Phi\right)}^{-1}} \cancel{\left(\Phi^{\top} D_{\mu} \Phi\right)} \theta_{*}
    =
    \Phi \theta_{*}
    \\
    \therefore
    \Pi T_{\mu} \Phi \theta_{*}
     & =
    \Phi \theta_{*}
\end{align*}
as required, showing that \( \theta_{*} \) is the fixed point of the projected Bellman operator.

Starting with \( A \theta_{*} = b \), and substituting the definitions of \( A \) and \( b \), we have
\begin{align*}
    \implies
    \Phi^{\top} D_{\mu}\left(\mathbb{I}-\gamma P_{\mu}\right) \Phi \theta_{*}
     & =
    \Phi^{\top} D_{\mu} r_{\mu}
    \\
    \implies
    \Phi^{\top} D_{\mu} \Phi \theta_{*} - \gamma \Phi^{\top} D_{\mu} P_{\mu} \Phi \theta_{*}
     & =
    \Phi^{\top} D_{\mu} r_{\mu}
    \\
    \implies
    \Phi^{\top} D_{\mu} \Phi \theta_{*}
     & =
    \Phi^{\top} D_{\mu} r_{\mu} + \gamma \Phi^{\top} D_{\mu} P_{\mu} \Phi \theta_{*}
    \\ & =
    \Phi^{\top} D_{\mu} \left( r_{\mu} + \gamma P_{\mu} \Phi \theta_{*} \right)
    =
    \Phi^{\top} D_{\mu} T_{\mu} \Phi \theta_{*}
\end{align*}

